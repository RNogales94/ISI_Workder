\input{preambuloSimple.tex}

%----------------------------------------------------------------------------------------
%	TÍTULO Y DATOS DEL ALUMNO
%----------------------------------------------------------------------------------------

\title{	
\normalfont \normalsize 
\textsc{\textbf{Ingeniería de Sistemas de Información (2017-2018)} \\ Grado en Ingeniería Informática \\ Universidad de Granada} \\ [25pt] % Your university, school and/or department name(s)
\horrule{0.5pt} \\[0.4cm] % Thin top horizontal rule
\huge Estudio preliminar \\ % The assignment title
\horrule{2pt} \\[0.5cm]
}

\author{Marta Amor Jurado \\ Melani Álvarez Santos \\ Rafa Nogales Vaquero} % Nombre y apellidos

\date{\normalsize\today} % Incluye la fecha actual

%----------------------------------------------------------------------------------------
% DOCUMENTO
%----------------------------------------------------------------------------------------

\begin{document}
\maketitle
\newpage 
\tableofcontents
\newpage
\setlength{\parindent}{1cm}

%----------------------------------------------------------------------------------------
%	Título provisionaldel proyecto
%----------------------------------------------------------------------------------------

\section{Título provisional del proyecto}

\begin{center}
	\textbf{Buscador de trabajo y trabajadores}
\end{center}


%----------------------------------------------------------------------------------------
%	Descripción del ámbito del proyecto
%----------------------------------------------------------------------------------------

\section{Descripción del ámbito del proyecto}
Hoy en día es difícil encontrar un trabajo que se adapte a nuestro objetivo o un trabajador que de un determinado perfil. Nuestro proyecto intentará acercar empresas y solicitantes de empleo de una manera fácil y rápida.
Tendremos dos perfiles claramente diferenciados, unos serán los que ofertan un trabajo y los otros aquellos que lo demandan. Cada uno de ellos deberá tener unas ciertas cualidades que lo harán diferenciable e identificarán con respecto a los demás. 

Se comparan las diferentes empresas para poder mostrarlas, posteriormente, en orden de mayor probabilidad de selección para un candidato en particular; de igual manera con los candidatos para una determinada empresa.

Por tanto, un usuario de la página web no se encarga de comparar dos empresas o dos candidatos (en el caso de formar parte de una empresa) en concreto, sino que, atendiendo a sus prioridades, le aparece un listado anteriormente comparado.

%----------------------------------------------------------------------------------------
%	Proyectos y sistemas existentes en el mismo ámbito
%----------------------------------------------------------------------------------------

\section{Proyectos y sistemas existentes en el mismo ámbito}
Los proyectos existentes que podemos encontrar en el mismo ámbito son los mismos de los que pretendemos extraer los datos para poder realizar nuestro proyecto. \\

Infojobs: Es un portal web en el que pymes y grandes empresas dan a conocer sus ofertas de trabajo y contratación. A su vez, los demandantes de empleo pueden avistar y seleccionar aquellas ofertas que les sean convenientes según su perfil profesional. \\

Linkedin: Es una red social para el ámbito profesional con un objetivo común: Cada usuario crea su propia red de contactos. Además, se encuentran en continuo contacto con el mundo empresarial y sus directivos gracias a la gran interacción entre los usuarios.
%----------------------------------------------------------------------------------------
%	Lista de fuentes de datos que se pretenden utilizar
%----------------------------------------------------------------------------------------
	
\section{Lista de fuentes de datos que se pretenden utilizar}
\begin{itemize}
	\item http://www.infojobs.com
	\item https://es.linkedin.com
	\item https://www.glassdoor.com
\end{itemize}	
	
\end{document}
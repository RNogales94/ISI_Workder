\input{preambuloSimple.tex}

%----------------------------------------------------------------------------------------
%	TÍTULO Y DATOS DEL ALUMNO
%----------------------------------------------------------------------------------------

\title{	
\normalfont \normalsize 
\textsc{\textbf{Ingeniería de Sistemas de Información (2017-2018)} \\ Grado en Ingeniería Informática \\ Universidad de Granada} \\ [25pt] % Your university, school and/or department name(s)
\horrule{0.5pt} \\[0.4cm] % Thin top horizontal rule
\huge Presupuesto de desarrollo del sistema \\ % The assignment title
\horrule{2pt} \\[0.5cm]
}

\author{Marta Amor Jurado \\ Melani Álvarez Santos \\ Rafa Nogales Vaquero} % Nombre y apellidos

\date{\normalsize\today} % Incluye la fecha actual

%----------------------------------------------------------------------------------------
% DOCUMENTO
%----------------------------------------------------------------------------------------

\begin{document}
\maketitle
\newpage 
\tableofcontents
\newpage
\setlength{\parindent}{1cm}

%----------------------------------------------------------------------------------------
%	Introducción
%----------------------------------------------------------------------------------------

\section{Introducción}
Partiendo de la propuesta del proyecto \textbf{Buscador de trabajo y trabajdores}, se realiza el presupuesto de desarrollo del mismo. El proyecto se llevará a cabo por tres expertos durante un período de tres meses.


%----------------------------------------------------------------------------------------
%	Gastos personales
%----------------------------------------------------------------------------------------

\section{Gastos personales}
El coste en gasto de personal será de 30.000\euro:
\begin{itemize}
	\item 3 personas x 3.000\euro/mes = 10000\euro.
	\item 10.000\euro \hspace{1pt} x 3 meses = 30.000\euro.
\end{itemize} 

%----------------------------------------------------------------------------------------
%	Gastos de ejecución
%----------------------------------------------------------------------------------------

\section{Gastos de ejecución}
\begin{itemize}
	\item \textbf{Costes de adquisición de material inventariable:}
		\begin{itemize}
			\item Equipos informáticos: (7\% de (1.500x3)) = 0.07 x 4.500 = 315\euro.
			\item Discos duros externos: 250\euro.
			\item Impresora: (4\% de 190) = 7.60\euro.
		\end{itemize}
	\item \textbf{Costes de adquisición de material fungible:}
		\begin{itemize}
			\item Material oficina (tóner, folios, impresos): 50\euro.
			\item Desgaste oficina: 60\euro.
			\item Gastos oficina (luz, conexión internet, alquiler, teléfono): 1.80\euro.
		\end{itemize}
	\item \textbf{Costes en contratos, patentes, licencias, consultoría y suministros:} 0\euro.	
\end{itemize}

%----------------------------------------------------------------------------------------
%	Gastos complementarios
%----------------------------------------------------------------------------------------
	
\section{Gastos complementarios}
\begin{itemize}
	\item Gastos de desplazamiento, viajes, estancia y dietas: 1.500\euro,\hspace{1pt} suponiendo unos 10 desplazamientos para reunirnos con el cliente.
	\item Gastos de material de difusión, ppublicaciones, promoción, catálagos, folletos cartelería, ect: 2500\euro \hspace{1pt}. Nos encargaremos de la promoción y difusión en redes sociales del proyecto realizado. Añadimos también un posicionamiento cuando se busque los términos relacionados con el trabajo apra que se muestren en los primeros puestos (1250\euro).
	\item Gastos de inscripción en cursos, congresos y seminarios relacionados con el proyecto: 0\euro.
\end{itemize}


%----------------------------------------------------------------------------------------
%	Presupuesto total
%----------------------------------------------------------------------------------------

\section{Presupuesto total}
Presupuesto total: 30.000 + 315 + 250 + 7.60 + 50 + 60 + 1.800 + 1.500 + 2.500 + 1.250 = 37732.6\euro.

\end{document}
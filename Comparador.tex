%%%%%%%%%%%%%%%%%%%%%%%%%%%%%%%%%%%%%%%%%
% Short Sectioned Assignment LaTeX Template Version 1.0 (5/5/12)
% This template has been downloaded from: http://www.LaTeXTemplates.com
% Original author:  Frits Wenneker (http://www.howtotex.com)
% License: CC BY-NC-SA 3.0 (http://creativecommons.org/licenses/by-nc-sa/3.0/)
%%%%%%%%%%%%%%%%%%%%%%%%%%%%%%%%%%%%%%%%%

%----------------------------------------------------------------------------------------
%	PACKAGES AND OTHER DOCUMENT CONFIGURATIONS
%----------------------------------------------------------------------------------------

\documentclass[paper=a4, fontsize=11pt]{scrartcl} % A4 paper and 11pt font size

% ---- Entrada y salida de texto -----

\usepackage[T1]{fontenc} % Use 8-bit encoding that has 256 glyphs
\usepackage[utf8]{inputenc}
%\usepackage{fourier} % Use the Adobe Utopia font for the document - comment this line to return to the LaTeX default

% ---- Idioma --------

\usepackage[spanish, es-tabla]{babel} % Selecciona el español para palabras introducidas automáticamente, p.ej. "septiembre" en la fecha y especifica que se use la palabra Tabla en vez de Cuadro

% ---- Otros paquetes ----

%https://en.wikibooks.org/wiki/LaTeX/Hyperlinks#.5Curl
\usepackage[colorlinks=true,linkcolor=blue,citecolor=red, urlcolor=blue]{hyperref} % para poder poner referencias, url...

%https://en.wikibooks.org/wiki/LaTeX/Colors
\usepackage{color, colortbl}
\usepackage[first=0,last=9]{lcg} %Color tabla
\newcommand{\ra}{\rand0.\arabic{rand}} %Color tabla

\usepackage{url}

\usepackage{eurosym}

\usepackage{url} % ,href} %para incluir URLs e hipervínculos dentro del texto (aunque hay que instalar href)
\usepackage{amsmath,amsfonts,amsthm} % Math packages
%\usepackage{graphics,graphicx, floatrow} %para incluir imágenes y notas en las imágenes
\usepackage{graphics,graphicx, float} %para incluir imágenes y colocarlas

% Para hacer tablas comlejas
%\usepackage{multirow}
%\usepackage{threeparttable}

%\usepackage{sectsty} % Allows customizing section commands
%\allsectionsfont{\centering \normalfont\scshape} % Make all sections centered, the default font and small caps

\usepackage{fancyhdr} % Custom headers and footers
\pagestyle{fancyplain} % Makes all pages in the document conform to the custom headers and footers
\fancyhead{} % No page header - if you want one, create it in the same way as the footers below
\fancyfoot[L]{} % Empty left footer
\fancyfoot[C]{} % Empty center footer
\fancyfoot[R]{\thepage} % Page numbering for right footer
\renewcommand{\headrulewidth}{0pt} % Remove header underlines
\renewcommand{\footrulewidth}{0pt} % Remove footer underlines
\setlength{\headheight}{13.6pt} % Customize the height of the header

\numberwithin{equation}{section} % Number equations within sections (i.e. 1.1, 1.2, 2.1, 2.2 instead of 1, 2, 3, 4)
\numberwithin{figure}{section} % Number figures within sections (i.e. 1.1, 1.2, 2.1, 2.2 instead of 1, 2, 3, 4)
\numberwithin{table}{section} % Number tables within sections (i.e. 1.1, 1.2, 2.1, 2.2 instead of 1, 2, 3, 4)

\setlength\parindent{0pt} % Removes all indentation from paragraphs - comment this line for an assignment with lots of text

\newcommand{\horrule}[1]{\rule{\linewidth}{#1}} % Create horizontal rule command with 1 argument of height


%----------------------------------------------------------------------------------------
%	TÍTULO Y DATOS DEL ALUMNO
%----------------------------------------------------------------------------------------

\title{	
\normalfont \normalsize 
\textsc{\textbf{Ingeniería de Sistemas de Información (2017-2018)} \\ Grado en Ingeniería Informática \\ Universidad de Granada} \\ [25pt] % Your university, school and/or department name(s)
\horrule{0.5pt} \\[0.4cm] % Thin top horizontal rule
\huge Estudio preliminar \\ % The assignment title
\horrule{2pt} \\[0.5cm]
}

\author{Marta Amor Jurado \\ Melani Álvarez Santos \\ Rafa Nogales Vaquero} % Nombre y apellidos

\date{\normalsize\today} % Incluye la fecha actual

%----------------------------------------------------------------------------------------
% DOCUMENTO
%----------------------------------------------------------------------------------------

\begin{document}
\maketitle
\newpage 
\tableofcontents
\newpage
\setlength{\parindent}{1cm}

%----------------------------------------------------------------------------------------
%	Título provisionaldel proyecto
%----------------------------------------------------------------------------------------

\section{}

\begin{center}
	\textbf{Buscador de trabajo y trabajadores}
\end{center}


%----------------------------------------------------------------------------------------
%	Descripción del ámbito del proyecto
%----------------------------------------------------------------------------------------

\section{Descripción del ámbito del proyecto}
Hoy en día es difícil encontrar un trabajo que se adapte a nuestro objetivo o un trabajador que de un determinado perfil. Nuestro proyecto intentará acercar empresas y solicitantes de empleo de una manera fácil y rápida.
Tendremos dos perfiles claramente diferenciados, unos serán los que ofertan un trabajo y los otros aquellos que lo demandan. Cada uno de ellos deberá tener unas ciertas cualidades que lo harán diferenciable e identificarán con respecto a los demás. 

Se comparan las diferentes empresas para poder mostrarlas, posteriormente, en orden de mayor probabilidad de selección para un candidato en particular; de igual manera con los candidatos para una determinada empresa.

Por tanto, un usuario de la página web no se encarga de comparar dos empresas o dos candidatos (en el caso de formar parte de una empresa) en concreto, sino que, atendiendo a sus prioridades, le aparece un listado anteriormente comparado.

%----------------------------------------------------------------------------------------
%	Proyectos y sistemas existentes en el mismo ámbito
%----------------------------------------------------------------------------------------

\section{Proyectos y sistemas existentes en el mismo ámbito}
Los proyectos existentes que podemos encontrar en el mismo ámbito son los mismos de los que pretendemos extraer los datos para poder realizar nuestro proyecto. \\

Infojobs: Es un portal web en el que pymes y grandes empresas dan a conocer sus ofertas de trabajo y contratación. A su vez, los demandantes de empleo pueden avistar y seleccionar aquellas ofertas que les sean convenientes según su perfil profesional. \\

Linkedin: Es una red social para el ámbito profesional con un objetivo común: Cada usuario crea su propia red de contactos. Además, se encuentran en continuo contacto con el mundo empresarial y sus directivos gracias a la gran interacción entre los usuarios.
%----------------------------------------------------------------------------------------
%	Lista de fuentes de datos que se pretenden utilizar
%----------------------------------------------------------------------------------------
	
\section{Lista de fuentes de datos que se pretenden utilizar}
\begin{itemize}
	\item http://www.infojobs.com
	\item https://es.linkedin.com
	\item https://www.glassdoor.com
\end{itemize}	
	
\end{document}

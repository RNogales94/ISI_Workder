%%%%%%%%%%%%%%%%%%%%%%%%%%%%%%%%%%%%%%%%%
% Short Sectioned Assignment LaTeX Template Version 1.0 (5/5/12)
% This template has been downloaded from: http://www.LaTeXTemplates.com
% Original author:  Frits Wenneker (http://www.howtotex.com)
% License: CC BY-NC-SA 3.0 (http://creativecommons.org/licenses/by-nc-sa/3.0/)
%%%%%%%%%%%%%%%%%%%%%%%%%%%%%%%%%%%%%%%%%

%----------------------------------------------------------------------------------------
%	PACKAGES AND OTHER DOCUMENT CONFIGURATIONS
%----------------------------------------------------------------------------------------

\documentclass[paper=a4, fontsize=11pt]{scrartcl} % A4 paper and 11pt font size

% ---- Entrada y salida de texto -----

\usepackage[T1]{fontenc} % Use 8-bit encoding that has 256 glyphs
\usepackage[utf8]{inputenc}
%\usepackage{fourier} % Use the Adobe Utopia font for the document - comment this line to return to the LaTeX default

% ---- Idioma --------

\usepackage[spanish, es-tabla]{babel} % Selecciona el español para palabras introducidas automáticamente, p.ej. "septiembre" en la fecha y especifica que se use la palabra Tabla en vez de Cuadro

% ---- Otros paquetes ----

%https://en.wikibooks.org/wiki/LaTeX/Hyperlinks#.5Curl
\usepackage[colorlinks=true,linkcolor=blue,citecolor=red, urlcolor=blue]{hyperref} % para poder poner referencias, url...

%https://en.wikibooks.org/wiki/LaTeX/Colors
\usepackage{color, colortbl}
\usepackage[first=0,last=9]{lcg} %Color tabla
\newcommand{\ra}{\rand0.\arabic{rand}} %Color tabla

\usepackage{url}

\usepackage{eurosym}

\usepackage{url} % ,href} %para incluir URLs e hipervínculos dentro del texto (aunque hay que instalar href)
\usepackage{amsmath,amsfonts,amsthm} % Math packages
%\usepackage{graphics,graphicx, floatrow} %para incluir imágenes y notas en las imágenes
\usepackage{graphics,graphicx, float} %para incluir imágenes y colocarlas

% Para hacer tablas comlejas
%\usepackage{multirow}
%\usepackage{threeparttable}

%\usepackage{sectsty} % Allows customizing section commands
%\allsectionsfont{\centering \normalfont\scshape} % Make all sections centered, the default font and small caps

\usepackage{fancyhdr} % Custom headers and footers
\pagestyle{fancyplain} % Makes all pages in the document conform to the custom headers and footers
\fancyhead{} % No page header - if you want one, create it in the same way as the footers below
\fancyfoot[L]{} % Empty left footer
\fancyfoot[C]{} % Empty center footer
\fancyfoot[R]{\thepage} % Page numbering for right footer
\renewcommand{\headrulewidth}{0pt} % Remove header underlines
\renewcommand{\footrulewidth}{0pt} % Remove footer underlines
\setlength{\headheight}{13.6pt} % Customize the height of the header

\numberwithin{equation}{section} % Number equations within sections (i.e. 1.1, 1.2, 2.1, 2.2 instead of 1, 2, 3, 4)
\numberwithin{figure}{section} % Number figures within sections (i.e. 1.1, 1.2, 2.1, 2.2 instead of 1, 2, 3, 4)
\numberwithin{table}{section} % Number tables within sections (i.e. 1.1, 1.2, 2.1, 2.2 instead of 1, 2, 3, 4)

\setlength\parindent{0pt} % Removes all indentation from paragraphs - comment this line for an assignment with lots of text

\newcommand{\horrule}[1]{\rule{\linewidth}{#1}} % Create horizontal rule command with 1 argument of height


%----------------------------------------------------------------------------------------
%	TÍTULO Y DATOS DEL ALUMNO
%----------------------------------------------------------------------------------------

\title{	
\normalfont \normalsize 
\textsc{\textbf{Ingeniería de Sistemas de Información (2017-2018)} \\ Grado en Ingeniería Informática \\ Universidad de Granada} \\ [25pt] % Your university, school and/or department name(s)
\horrule{0.5pt} \\[0.4cm] % Thin top horizontal rule
\huge Presupuesto de desarrollo del sistema \\ % The assignment title
\horrule{2pt} \\[0.5cm]
}

\author{Marta Amor Jurado \\ Melani Álvarez Santos \\ Rafa Nogales Vaquero} % Nombre y apellidos

\date{\normalsize\today} % Incluye la fecha actual

%----------------------------------------------------------------------------------------
% DOCUMENTO
%----------------------------------------------------------------------------------------

\begin{document}
\maketitle
\newpage 
\tableofcontents
\newpage
\setlength{\parindent}{1cm}

%----------------------------------------------------------------------------------------
%	Introducción
%----------------------------------------------------------------------------------------

\section{Introducción}
Partiendo de la propuesta del proyecto \textbf{Buscador de trabajo y trabajdores}, se realiza el presupuesto de desarrollo del mismo. El proyecto se llevará a cabo por tres expertos durante un período de tres meses.


%----------------------------------------------------------------------------------------
%	Gastos personales
%----------------------------------------------------------------------------------------

\section{Gastos personales}
El coste en gasto de personal será de 30.000\euro:
\begin{itemize}
	\item 3 personas x 3.000\euro/mes = 10000\euro.
	\item 10.000\euro \hspace{1pt} x 3 meses = 30.000\euro.
\end{itemize} 

%----------------------------------------------------------------------------------------
%	Gastos de ejecución
%----------------------------------------------------------------------------------------

\section{Gastos de ejecución}
\begin{itemize}
	\item \textbf{Costes de adquisición de material inventariable:}
		\begin{itemize}
			\item Equipos informáticos: (7\% de (1.500x3)) = 0.07 x 4.500 = 315\euro.
			\item Discos duros externos: 250\euro.
			\item Impresora: (4\% de 190) = 7.60\euro.
		\end{itemize}
	\item \textbf{Costes de adquisición de material fungible:}
		\begin{itemize}
			\item Material oficina (tóner, folios, impresos): 50\euro.
			\item Desgaste oficina: 60\euro.
			\item Gastos oficina (luz, conexión internet, alquiler, teléfono): 1.80\euro.
		\end{itemize}
	\item \textbf{Costes en contratos, patentes, licencias, consultoría y suministros:} 0\euro.	
\end{itemize}

%----------------------------------------------------------------------------------------
%	Gastos complementarios
%----------------------------------------------------------------------------------------
	
\section{Gastos complementarios}
\begin{itemize}
	\item Gastos de desplazamiento, viajes, estancia y dietas: 1.500\euro,\hspace{1pt} suponiendo unos 10 desplazamientos para reunirnos con el cliente.
	\item Gastos de material de difusión, ppublicaciones, promoción, catálagos, folletos cartelería, ect: 2500\euro \hspace{1pt}. Nos encargaremos de la promoción y difusión en redes sociales del proyecto realizado. Añadimos también un posicionamiento cuando se busque los términos relacionados con el trabajo apra que se muestren en los primeros puestos (1250\euro).
	\item Gastos de inscripción en cursos, congresos y seminarios relacionados con el proyecto: 0\euro.
\end{itemize}


%----------------------------------------------------------------------------------------
%	Presupuesto total
%----------------------------------------------------------------------------------------

\section{Presupuesto total}
Presupuesto total: 30.000 + 315 + 250 + 7.60 + 50 + 60 + 1.800 + 1.500 + 2.500 + 1.250 = 37732.6\euro.

\end{document}